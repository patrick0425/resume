% LaTeX file for resume 
% This file uses the resume document class (res.cls)

\documentclass[margin]{res} 
% the margin option causes section titles to appear to the left of body text 
\oddsidemargin -.1in
\evensidemargin -.1in
\textwidth=5.1in % increase textwidth to get smaller right margin
\setlength{\textheight}{9in}
%\usepackage{helvetica} % uses helvetica postscript font (download helvetica.sty)
\usepackage{newcent}   % uses new century schoolbook postscript font
%\usepackage{pxfonts}
%\addtolength{\textheight}{0.1in}

\begin{document} 

% Center the name over the entire width of resume:
 \moveleft.5\hoffset\centerline{\Large\bf Xiang Ji}
% Draw a horizontal line the whole width of resume:
 \moveleft\hoffset\vbox{\hrule width\resumewidth height 1pt}\smallskip
% address begins here
% Again, the address lines must be centered over entire width of resume:
 \moveleft.5\hoffset\centerline{jxpatpat@gmail.com}
 \moveleft.5\hoffset\centerline{(519) 505-7338}
 \moveleft.5\hoffset\centerline{7 Cardill Crescent, Waterloo, Ontario, Canada N2L3Y7}




\begin{resume} 
 
\section{Objective}

{\bf Software Engineer, Machine Learning}



\section{Qualifications}

{\bf Programming languages}\\
Java, Python, C, C++, Scala, Scheme 

{\bf Developing tools}\\
{\it Machine learning:} Mahout, MATLAB\\
{\it Distributed system:} Hadoop, Pig, Scalding, Storm, Summingbird\\
{\it Cross-protocol development:} Thrift, Finagle\\
{\it Efficiency tooling:} Intellij IDEA, Eclipse, Mesos, Maven, Ant\\
{\it Neural network simulation:} Nengo, NuPic

{\bf Experienced fields}
\begin{itemize} \itemsep -2pt
\item Neural network modeling 
\item Large scale machine learning system design / implementation
\item Algorithm and data structure
\item Basic knowledge in security, OS, UI, computer vision, etc.
\end{itemize}

 

\section{Education}

 {\bf Master of Mathematics, Computer Science} \hfill 2012.4 -- 2014.4 (expected)\\
 University of Waterloo, Waterloo, Canada\\
 {\bf Thesis topic:} Path Integration with Velocity-Controlled Oscillators\\
 {\bf Relevant courses:} Computational Neuroscience,  Applied Machine Learning,\\
 Probabilistic Inference and Machine Learning
 
 {\bf Exchange Student, Computer Science} \hfill 2011.9 -- 2012.3\\
 University of Waterloo, Waterloo, Canada\\
 {\bf Thesis topic:} Hippocampus Modeling on Spatial Alternation Task\\
 {\bf Relevant courses:} User Interfaces, Machine Learning, Algorithms, \\
 Computer Vision
 
 {\bf Bachelor of Engineering, Computer Science} \hfill 2008.9 -- 2012.6\\
 Tsinghua University, Beijing, China\\
 {\bf Relevant courses:} Artificial Intelligence, Operating System, Network, \\
 Computer Architecture, Data Structures

 
 
 \section{Internship}

 {\bf SDE -- Twitter Inc., San Francisco} \hfill 2013.8 -- 2013.12 (expected)\\
 Developing a large scale real-time recommendation system infrastructure
  \begin{itemize} \itemsep -2pt
  \item Serving most of Twitter's recommendation products
  \item Using content-boosted collaborative filtering with random walk\\ model on Hadoop / Storm
  \end{itemize}

 {\bf SDET -- Hulu LLC., Beijing} \hfill 2010.9 -- 2010.12\\
 Developed recommendation system unit tests
  \begin{itemize} \itemsep -2pt
  \item Implemented Automatic testing in Ruby and Java
  \item Deployed test coverage tool Emma for java tests
  \end{itemize} 



\section{Projects} 

 {\bf Modeling Path Integration using Velocity Controlled Oscillators}\\
 Computational Neuroscience
 \begin{itemize}  \itemsep -2pt
  \item Simulated rat's hippocampus using {\raise.17ex\hbox{$\scriptstyle\sim$}}50,000 virtual neurons
  \item Built a virtual rat that is able to navigate in a 2D space
  \item Included stabilizing mechanisms and sensory inputs
 \end{itemize}
 
 {\bf Multi-level Position Reconstruction from Hippocampal Place Cells}\\
 Applied Machine Learning
 \begin{itemize}  \itemsep -2pt
  \item Implemented machine learning algorithms on {\raise.17ex\hbox{$\scriptstyle\sim$}}20GB neural data
  \item Designed multiple feature levels for faster and more accurate learning
  \item Involved Bayesian networks in learning on neural data
  \item Average error reduced to 1/3 of previous results
 \end{itemize}
 
 {\bf Private Learning with Homomorphic Encryption}\\
 Probabilistic Inference and Machine Learning
  \begin{itemize}  \itemsep -2pt
  \item Reviewed different private machine learning approaches
  \item Discussed the difference of schemes and algorithms
  \item Evaluated algorithm efficiency based on feature amount and data size
 \end{itemize}
 
  {\bf Approaches to Handwritten Digit Recognition}\\
 Machine Learning
  \begin{itemize}  \itemsep -2pt
  \item Implemented several ML algorithms on recognizing handwritten digits
  \item Compared time and accuracy of logistic regression, SVM and ANN
 \end{itemize}
 
 {\bf Talking Avatar with Facial Expressions on Android Platform}\\
 Summer workshop
  \begin{itemize} \itemsep -2pt
   \item Built a virtual face with expressions and voice on Android platform
   \item Involved in expression modeling, audio-video sync, UI design, etc.
   \item Used Java and C, including JNI
 \end{itemize}



\section{Publications} 

 {\bf Articles in Refereed Journals}\\
 \begin{enumerate} \itemsep -2pt
\item {\bf X. Ji}, S. Kushagra, J. Orchard, "Updating the Entorhinal Cortex Fourier Model with Visual-Sensory Input", {\em Canadian Conference on Artificial Intelligence (AI) 2013}.
\item J. Orchard, H Yang, {\bf X. Ji}, "Does the Entorhinal Cortex use the Fourier Transform?",  {\em Canadian Conference on Artificial Intelligence (AI) 2013}.
\item B. Liu, G. Wu, Z. Wang, {\bf X. Ji}, ``Semantic integration of differently asynchronous audio–visual information in videos of real-world events in cognitive processing: An ERP study'', {\em Neuroscience Letters}, July 2011. 
\end{enumerate}


\section{Awards}

David R. Cheriton Graduate Scholarship, \$10,000 \hfill 2012 -- 2013\\
UW Special Graduate Scholarship, \$4,000 \hfill 2012 -- 2013\\
Outstanding Student Leader, Tsinghua University \hfill 2011\\
Tencent Scholarship, RMB 1,000 \hfill 2009\\
National Physics Competition for University Students, Second Prize \hfill 2009\\
National Physics Olympiad, First Prize \hfill 2008\\



\section{Interests}
{\bf Machine learning:} Deep learning; Large scale / parallel / online ML schemes\\
{\bf Brain simulation:} How can bottom-up methods meet top-down methods\\
{\bf Others:} Music arrangement; Photography; Jogging; Cycling; Gaming\\

\end{resume} 
\end{document} 



